% Options for packages loaded elsewhere
\PassOptionsToPackage{unicode}{hyperref}
\PassOptionsToPackage{hyphens}{url}
\PassOptionsToPackage{dvipsnames,svgnames,x11names}{xcolor}
%
\documentclass[
  letterpaper,
  DIV=11,
  numbers=noendperiod]{scrartcl}

\usepackage{amsmath,amssymb}
\usepackage{iftex}
\ifPDFTeX
  \usepackage[T1]{fontenc}
  \usepackage[utf8]{inputenc}
  \usepackage{textcomp} % provide euro and other symbols
\else % if luatex or xetex
  \usepackage{unicode-math}
  \defaultfontfeatures{Scale=MatchLowercase}
  \defaultfontfeatures[\rmfamily]{Ligatures=TeX,Scale=1}
\fi
\usepackage{lmodern}
\ifPDFTeX\else  
    % xetex/luatex font selection
\fi
% Use upquote if available, for straight quotes in verbatim environments
\IfFileExists{upquote.sty}{\usepackage{upquote}}{}
\IfFileExists{microtype.sty}{% use microtype if available
  \usepackage[]{microtype}
  \UseMicrotypeSet[protrusion]{basicmath} % disable protrusion for tt fonts
}{}
\makeatletter
\@ifundefined{KOMAClassName}{% if non-KOMA class
  \IfFileExists{parskip.sty}{%
    \usepackage{parskip}
  }{% else
    \setlength{\parindent}{0pt}
    \setlength{\parskip}{6pt plus 2pt minus 1pt}}
}{% if KOMA class
  \KOMAoptions{parskip=half}}
\makeatother
\usepackage{xcolor}
\setlength{\emergencystretch}{3em} % prevent overfull lines
\setcounter{secnumdepth}{-\maxdimen} % remove section numbering
% Make \paragraph and \subparagraph free-standing
\ifx\paragraph\undefined\else
  \let\oldparagraph\paragraph
  \renewcommand{\paragraph}[1]{\oldparagraph{#1}\mbox{}}
\fi
\ifx\subparagraph\undefined\else
  \let\oldsubparagraph\subparagraph
  \renewcommand{\subparagraph}[1]{\oldsubparagraph{#1}\mbox{}}
\fi

\usepackage{color}
\usepackage{fancyvrb}
\newcommand{\VerbBar}{|}
\newcommand{\VERB}{\Verb[commandchars=\\\{\}]}
\DefineVerbatimEnvironment{Highlighting}{Verbatim}{commandchars=\\\{\}}
% Add ',fontsize=\small' for more characters per line
\usepackage{framed}
\definecolor{shadecolor}{RGB}{241,243,245}
\newenvironment{Shaded}{\begin{snugshade}}{\end{snugshade}}
\newcommand{\AlertTok}[1]{\textcolor[rgb]{0.68,0.00,0.00}{#1}}
\newcommand{\AnnotationTok}[1]{\textcolor[rgb]{0.37,0.37,0.37}{#1}}
\newcommand{\AttributeTok}[1]{\textcolor[rgb]{0.40,0.45,0.13}{#1}}
\newcommand{\BaseNTok}[1]{\textcolor[rgb]{0.68,0.00,0.00}{#1}}
\newcommand{\BuiltInTok}[1]{\textcolor[rgb]{0.00,0.23,0.31}{#1}}
\newcommand{\CharTok}[1]{\textcolor[rgb]{0.13,0.47,0.30}{#1}}
\newcommand{\CommentTok}[1]{\textcolor[rgb]{0.37,0.37,0.37}{#1}}
\newcommand{\CommentVarTok}[1]{\textcolor[rgb]{0.37,0.37,0.37}{\textit{#1}}}
\newcommand{\ConstantTok}[1]{\textcolor[rgb]{0.56,0.35,0.01}{#1}}
\newcommand{\ControlFlowTok}[1]{\textcolor[rgb]{0.00,0.23,0.31}{#1}}
\newcommand{\DataTypeTok}[1]{\textcolor[rgb]{0.68,0.00,0.00}{#1}}
\newcommand{\DecValTok}[1]{\textcolor[rgb]{0.68,0.00,0.00}{#1}}
\newcommand{\DocumentationTok}[1]{\textcolor[rgb]{0.37,0.37,0.37}{\textit{#1}}}
\newcommand{\ErrorTok}[1]{\textcolor[rgb]{0.68,0.00,0.00}{#1}}
\newcommand{\ExtensionTok}[1]{\textcolor[rgb]{0.00,0.23,0.31}{#1}}
\newcommand{\FloatTok}[1]{\textcolor[rgb]{0.68,0.00,0.00}{#1}}
\newcommand{\FunctionTok}[1]{\textcolor[rgb]{0.28,0.35,0.67}{#1}}
\newcommand{\ImportTok}[1]{\textcolor[rgb]{0.00,0.46,0.62}{#1}}
\newcommand{\InformationTok}[1]{\textcolor[rgb]{0.37,0.37,0.37}{#1}}
\newcommand{\KeywordTok}[1]{\textcolor[rgb]{0.00,0.23,0.31}{#1}}
\newcommand{\NormalTok}[1]{\textcolor[rgb]{0.00,0.23,0.31}{#1}}
\newcommand{\OperatorTok}[1]{\textcolor[rgb]{0.37,0.37,0.37}{#1}}
\newcommand{\OtherTok}[1]{\textcolor[rgb]{0.00,0.23,0.31}{#1}}
\newcommand{\PreprocessorTok}[1]{\textcolor[rgb]{0.68,0.00,0.00}{#1}}
\newcommand{\RegionMarkerTok}[1]{\textcolor[rgb]{0.00,0.23,0.31}{#1}}
\newcommand{\SpecialCharTok}[1]{\textcolor[rgb]{0.37,0.37,0.37}{#1}}
\newcommand{\SpecialStringTok}[1]{\textcolor[rgb]{0.13,0.47,0.30}{#1}}
\newcommand{\StringTok}[1]{\textcolor[rgb]{0.13,0.47,0.30}{#1}}
\newcommand{\VariableTok}[1]{\textcolor[rgb]{0.07,0.07,0.07}{#1}}
\newcommand{\VerbatimStringTok}[1]{\textcolor[rgb]{0.13,0.47,0.30}{#1}}
\newcommand{\WarningTok}[1]{\textcolor[rgb]{0.37,0.37,0.37}{\textit{#1}}}

\providecommand{\tightlist}{%
  \setlength{\itemsep}{0pt}\setlength{\parskip}{0pt}}\usepackage{longtable,booktabs,array}
\usepackage{calc} % for calculating minipage widths
% Correct order of tables after \paragraph or \subparagraph
\usepackage{etoolbox}
\makeatletter
\patchcmd\longtable{\par}{\if@noskipsec\mbox{}\fi\par}{}{}
\makeatother
% Allow footnotes in longtable head/foot
\IfFileExists{footnotehyper.sty}{\usepackage{footnotehyper}}{\usepackage{footnote}}
\makesavenoteenv{longtable}
\usepackage{graphicx}
\makeatletter
\def\maxwidth{\ifdim\Gin@nat@width>\linewidth\linewidth\else\Gin@nat@width\fi}
\def\maxheight{\ifdim\Gin@nat@height>\textheight\textheight\else\Gin@nat@height\fi}
\makeatother
% Scale images if necessary, so that they will not overflow the page
% margins by default, and it is still possible to overwrite the defaults
% using explicit options in \includegraphics[width, height, ...]{}
\setkeys{Gin}{width=\maxwidth,height=\maxheight,keepaspectratio}
% Set default figure placement to htbp
\makeatletter
\def\fps@figure{htbp}
\makeatother

\KOMAoption{captions}{tableheading}
\makeatletter
\makeatother
\makeatletter
\makeatother
\makeatletter
\@ifpackageloaded{caption}{}{\usepackage{caption}}
\AtBeginDocument{%
\ifdefined\contentsname
  \renewcommand*\contentsname{Table of contents}
\else
  \newcommand\contentsname{Table of contents}
\fi
\ifdefined\listfigurename
  \renewcommand*\listfigurename{List of Figures}
\else
  \newcommand\listfigurename{List of Figures}
\fi
\ifdefined\listtablename
  \renewcommand*\listtablename{List of Tables}
\else
  \newcommand\listtablename{List of Tables}
\fi
\ifdefined\figurename
  \renewcommand*\figurename{Figure}
\else
  \newcommand\figurename{Figure}
\fi
\ifdefined\tablename
  \renewcommand*\tablename{Table}
\else
  \newcommand\tablename{Table}
\fi
}
\@ifpackageloaded{float}{}{\usepackage{float}}
\floatstyle{ruled}
\@ifundefined{c@chapter}{\newfloat{codelisting}{h}{lop}}{\newfloat{codelisting}{h}{lop}[chapter]}
\floatname{codelisting}{Listing}
\newcommand*\listoflistings{\listof{codelisting}{List of Listings}}
\makeatother
\makeatletter
\@ifpackageloaded{caption}{}{\usepackage{caption}}
\@ifpackageloaded{subcaption}{}{\usepackage{subcaption}}
\makeatother
\makeatletter
\@ifpackageloaded{tcolorbox}{}{\usepackage[skins,breakable]{tcolorbox}}
\makeatother
\makeatletter
\@ifundefined{shadecolor}{\definecolor{shadecolor}{rgb}{.97, .97, .97}}
\makeatother
\makeatletter
\makeatother
\makeatletter
\makeatother
\ifLuaTeX
  \usepackage{selnolig}  % disable illegal ligatures
\fi
\IfFileExists{bookmark.sty}{\usepackage{bookmark}}{\usepackage{hyperref}}
\IfFileExists{xurl.sty}{\usepackage{xurl}}{} % add URL line breaks if available
\urlstyle{same} % disable monospaced font for URLs
\hypersetup{
  pdftitle={PyStats},
  pdfauthor={André van Zyl},
  colorlinks=true,
  linkcolor={blue},
  filecolor={Maroon},
  citecolor={Blue},
  urlcolor={Blue},
  pdfcreator={LaTeX via pandoc}}

\title{PyStats}
\author{André van Zyl}
\date{2023-11-24}

\begin{document}
\maketitle
\ifdefined\Shaded\renewenvironment{Shaded}{\begin{tcolorbox}[enhanced, boxrule=0pt, interior hidden, frame hidden, sharp corners, borderline west={3pt}{0pt}{shadecolor}, breakable]}{\end{tcolorbox}}\fi

\hypertarget{setup}{%
\section{Setup}\label{setup}}

\hypertarget{packages}{%
\subsection{Packages}\label{packages}}

\begin{Shaded}
\begin{Highlighting}[]
\ImportTok{import}\NormalTok{ warnings}
\NormalTok{warnings.filterwarnings(}\StringTok{\textquotesingle{}ignore\textquotesingle{}}\NormalTok{)}
\ImportTok{import}\NormalTok{ random}
\ImportTok{import}\NormalTok{ numpy }\ImportTok{as}\NormalTok{ np}
\ImportTok{import}\NormalTok{ pandas }\ImportTok{as}\NormalTok{ pd}
\ImportTok{import}\NormalTok{ matplotlib.pyplot }\ImportTok{as}\NormalTok{ plt}
\ImportTok{import}\NormalTok{ seaborn }\ImportTok{as}\NormalTok{ sns}
\ImportTok{from}\NormalTok{ IPython.display }\ImportTok{import}\NormalTok{ display, Math}
\ImportTok{from}\NormalTok{ sympy }\ImportTok{import}\NormalTok{ latex, symbols, diff}
\ImportTok{import}\NormalTok{ numpy }\ImportTok{as}\NormalTok{ np}
\NormalTok{pd.set\_option(}\StringTok{\textquotesingle{}display.precision\textquotesingle{}}\NormalTok{, }\DecValTok{2}\NormalTok{)}
\NormalTok{np.set\_printoptions(precision}\OperatorTok{=}\DecValTok{2}\NormalTok{)}
\NormalTok{random.seed(}\DecValTok{1738}\NormalTok{)}
\end{Highlighting}
\end{Shaded}

\hypertarget{functions}{%
\subsection{Functions}\label{functions}}

\begin{Shaded}
\begin{Highlighting}[]
\KeywordTok{def}\NormalTok{ round\_values(numbers, precision):}
    \CommentTok{"""}
\CommentTok{    Rounds a list of numbers to a given precision.}

\CommentTok{    Parameters:}
\CommentTok{    numbers (list of float): The list of numbers to round.}
\CommentTok{    precision (int): The number of decimal places to round each number to.}

\CommentTok{    Returns:}
\CommentTok{    list of float: The list of rounded numbers.}
\CommentTok{    """}
    \ControlFlowTok{return}\NormalTok{ [}\BuiltInTok{round}\NormalTok{(num, precision) }\ControlFlowTok{for}\NormalTok{ num }\KeywordTok{in}\NormalTok{ numbers]}
\end{Highlighting}
\end{Shaded}

\hypertarget{population-proportion}{%
\section{Population Proportion}\label{population-proportion}}

\hypertarget{one-proportion}{%
\subsection{One Proportion}\label{one-proportion}}

\hypertarget{estimating-population-proportion}{%
\subsubsection{Estimating Population
Proportion}\label{estimating-population-proportion}}

A sample of 659 parents with a toddler was taken and asked if they used
a car seat for all travel with their toddler. 540 parents responded
`yes' to this question. Calcualte the proportion and 95\% CIs

\hypertarget{normal-approximation-method}{%
\paragraph{Normal Approximation
Method}\label{normal-approximation-method}}

\[
\hat{p} \pm z \sqrt{\frac{\hat{p}(1 - \hat{p})}{n}}
\]

Breaking it down:

\begin{itemize}
\tightlist
\item
  \(\hat{p}\) represents the sample proportion.
\item
  \(z\) is the z-score corresponding to the confidence level (1.96 for
  95\% confidence).
\item
  \(n\) is the sample size.
\item
  The square root term \(\sqrt{\frac{\hat{p}(1 - \hat{p})}{n}}\)
  calculates the standard error of the sample proportion.
\end{itemize}

\begin{Shaded}
\begin{Highlighting}[]

\CommentTok{\# Given data}
\NormalTok{count }\OperatorTok{=} \DecValTok{540}  \CommentTok{\# number of parents who responded \textquotesingle{}yes\textquotesingle{}}
\NormalTok{nobs }\OperatorTok{=} \DecValTok{659}   \CommentTok{\# total number of parents asked}
\CommentTok{\# Calculate the proportion}
\NormalTok{proportion }\OperatorTok{=}\NormalTok{ count }\OperatorTok{/}\NormalTok{ nobs}
\CommentTok{\# Proportion as per the new information}
\NormalTok{p\_hat }\OperatorTok{=}\NormalTok{ proportion}
\CommentTok{\# Total number of parents surveyed}
\NormalTok{n }\OperatorTok{=}\NormalTok{ nobs}
\CommentTok{\# Z{-}score for 95\% confidence}
\NormalTok{z }\OperatorTok{=} \FloatTok{1.96}

\CommentTok{\# Calculate the standard error}
\NormalTok{se }\OperatorTok{=}\NormalTok{ np.sqrt(p\_hat }\OperatorTok{*}\NormalTok{ (}\DecValTok{1} \OperatorTok{{-}}\NormalTok{ p\_hat) }\OperatorTok{/}\NormalTok{ n)}

\CommentTok{\# Calculate the 95\% confidence interval using the given formula}
\NormalTok{ci\_lower }\OperatorTok{=}\NormalTok{ p\_hat }\OperatorTok{{-}}\NormalTok{ z }\OperatorTok{*}\NormalTok{ se}
\NormalTok{ci\_upper }\OperatorTok{=}\NormalTok{ p\_hat }\OperatorTok{+}\NormalTok{ z }\OperatorTok{*}\NormalTok{ se}

\NormalTok{round\_values([p\_hat, ci\_lower, ci\_upper], }\DecValTok{2}\NormalTok{)}
\end{Highlighting}
\end{Shaded}

\begin{verbatim}
[0.82, 0.79, 0.85]
\end{verbatim}

\hypertarget{exact-binomial-method}{%
\paragraph{Exact (Binomial) Method}\label{exact-binomial-method}}

\begin{Shaded}
\begin{Highlighting}[]
\ImportTok{from}\NormalTok{ statsmodels.stats.proportion }\ImportTok{import}\NormalTok{ proportion\_confint}

\CommentTok{\# Calculate the 95\% confidence interval for the proportion}
\CommentTok{\# Method \textquotesingle{}binom\_test\textquotesingle{} is used for exact CI calculation appropriate for binomial distribution}
\NormalTok{ci\_low, ci\_upp }\OperatorTok{=}\NormalTok{ proportion\_confint(count, nobs, alpha}\OperatorTok{=}\FloatTok{0.05}\NormalTok{, method}\OperatorTok{=}\StringTok{\textquotesingle{}binom\_test\textquotesingle{}}\NormalTok{)}

\NormalTok{round\_values([proportion, ci\_low, ci\_upp], }\DecValTok{2}\NormalTok{)}
\end{Highlighting}
\end{Shaded}

\begin{verbatim}
[0.82, 0.79, 0.85]
\end{verbatim}

\hypertarget{sample-size-determination}{%
\subsubsection{Sample Size
Determination}\label{sample-size-determination}}

\hypertarget{normal-approximation-method-1}{%
\paragraph{Normal Approximation
Method}\label{normal-approximation-method-1}}

\[MoE = \frac{{1}}{\sqrt{n}}\]

\begin{Shaded}
\begin{Highlighting}[]
\NormalTok{moe }\OperatorTok{=} \DecValTok{1}\OperatorTok{/}\NormalTok{np.sqrt(}\DecValTok{232}\NormalTok{)}
\NormalTok{moe}
\end{Highlighting}
\end{Shaded}

\begin{verbatim}
0.06565321642986127
\end{verbatim}

\hypertarget{for-95-confidence}{%
\subparagraph{For 95\% Confidence}\label{for-95-confidence}}

\begin{Shaded}
\begin{Highlighting}[]
\CommentTok{\# Margin of Error (MoE)}
\NormalTok{MoE }\OperatorTok{=} \FloatTok{0.03}

\CommentTok{\# Calculate the sample size using the simplified formula from the image}
\CommentTok{\# n = (1 / MoE)\^{}2}
\NormalTok{sample\_size }\OperatorTok{=}\NormalTok{ (}\DecValTok{1} \OperatorTok{/}\NormalTok{ MoE)}\OperatorTok{**}\DecValTok{2}

\CommentTok{\# The sample size should be a whole number}
\NormalTok{sample\_size }\OperatorTok{=} \BuiltInTok{int}\NormalTok{(}\BuiltInTok{round}\NormalTok{(sample\_size))}

\NormalTok{sample\_size}
\end{Highlighting}
\end{Shaded}

\begin{verbatim}
1111
\end{verbatim}

\begin{Shaded}
\begin{Highlighting}[]
\CommentTok{\# Margin of Error (MoE)}
\NormalTok{MoE }\OperatorTok{=} \FloatTok{0.03}

\CommentTok{\# Calculate the sample size using the simplified formula from the image}
\CommentTok{\# n = (1 / MoE)\^{}2}
\NormalTok{sample\_size }\OperatorTok{=}\NormalTok{ (}\DecValTok{1} \OperatorTok{/}\NormalTok{ MoE)}\OperatorTok{**}\DecValTok{2}

\CommentTok{\# The sample size should be a whole number}
\NormalTok{sample\_size }\OperatorTok{=} \BuiltInTok{int}\NormalTok{(}\BuiltInTok{round}\NormalTok{(sample\_size))}

\NormalTok{sample\_size}
\end{Highlighting}
\end{Shaded}

\begin{verbatim}
1111
\end{verbatim}

\hypertarget{for-99-confidence}{%
\subparagraph{For 99\% Confidence}\label{for-99-confidence}}

\[
\begin{align*}
& \hat{p} \pm Z^* \cdot \frac{1}{2\sqrt{n}} \\
& MoE = Z^* \cdot \frac{1}{2\sqrt{n}} \\
& n = \left( \frac{Z^*}{2 \cdot MoE} \right)^2 \\
\end{align*}
\]

\begin{Shaded}
\begin{Highlighting}[]
\CommentTok{\# Values provided in the image}
\NormalTok{Z\_star }\OperatorTok{=} \FloatTok{2.576}  \CommentTok{\# Z{-}score for 99\% confidence}
\NormalTok{MoE }\OperatorTok{=} \FloatTok{0.03}      \CommentTok{\# Margin of Error}

\CommentTok{\# Sample size calculation}
\NormalTok{n }\OperatorTok{=}\NormalTok{ (Z\_star }\OperatorTok{/}\NormalTok{ (}\DecValTok{2} \OperatorTok{*}\NormalTok{ MoE)) }\OperatorTok{**} \DecValTok{2}

\CommentTok{\# Since sample size must be a whole number, round up}
\ImportTok{import}\NormalTok{ math}
\NormalTok{n\_rounded }\OperatorTok{=}\NormalTok{ math.ceil(n)}

\BuiltInTok{print}\NormalTok{(n\_rounded)}
\end{Highlighting}
\end{Shaded}

\begin{verbatim}
1844
\end{verbatim}

\hypertarget{exact-method}{%
\paragraph{Exact Method}\label{exact-method}}

\(n = \left( \frac{Z_{\alpha/2} \times \sqrt{p(1-p)}}{MoE} \right)^2\)

\begin{Shaded}
\begin{Highlighting}[]
\ImportTok{from}\NormalTok{ scipy.stats }\ImportTok{import}\NormalTok{ norm}

\CommentTok{\# Desired margin of error}
\NormalTok{margin\_of\_error }\OperatorTok{=} \FloatTok{0.04}

\CommentTok{\# For a 95\% confidence interval, the z{-}score is approximately 1.96}
\NormalTok{z\_score }\OperatorTok{=}\NormalTok{ norm.ppf(}\FloatTok{0.975}\NormalTok{)}

\CommentTok{\# Since the population proportion is unknown, we use the most conservative estimate, p = 0.5}
\NormalTok{p }\OperatorTok{=} \FloatTok{0.5}
\NormalTok{q }\OperatorTok{=} \DecValTok{1} \OperatorTok{{-}}\NormalTok{ p}

\CommentTok{\# Calculate the sample size using the formula for margin of error}
\NormalTok{sample\_size }\OperatorTok{=}\NormalTok{ (z\_score }\OperatorTok{**} \DecValTok{2} \OperatorTok{*}\NormalTok{ p }\OperatorTok{*}\NormalTok{ q) }\OperatorTok{/}\NormalTok{ (margin\_of\_error }\OperatorTok{**} \DecValTok{2}\NormalTok{)}

\CommentTok{\# Since sample size must be a whole number, we round up}
\NormalTok{sample\_size }\OperatorTok{=} \BuiltInTok{int}\NormalTok{(sample\_size) }\OperatorTok{+} \DecValTok{1} \ControlFlowTok{if}\NormalTok{ sample\_size }\OperatorTok{\%} \DecValTok{1} \OperatorTok{\textgreater{}} \DecValTok{0} \ControlFlowTok{else} \BuiltInTok{int}\NormalTok{(sample\_size)}

\NormalTok{sample\_size}
\end{Highlighting}
\end{Shaded}

\begin{verbatim}
601
\end{verbatim}

What minimum sample size does the researcher need in order to create a
98\% conservative confidence interval with a margin of error of no more
than 3\%?

\begin{Shaded}
\begin{Highlighting}[]
\ImportTok{from}\NormalTok{ scipy.stats }\ImportTok{import}\NormalTok{ norm}

\CommentTok{\# Desired margin of error}
\NormalTok{margin\_of\_error }\OperatorTok{=} \FloatTok{0.03}

\CommentTok{\# For a 98\% confidence interval, the z{-}score is approximately 2.33 (found using the ppf function)}
\NormalTok{z\_score }\OperatorTok{=}\NormalTok{ norm.ppf(}\FloatTok{0.99}\NormalTok{)}

\CommentTok{\# Since the population proportion is unknown, we use the most conservative estimate, p = 0.5}
\NormalTok{p }\OperatorTok{=} \FloatTok{0.5}
\NormalTok{q }\OperatorTok{=} \DecValTok{1} \OperatorTok{{-}}\NormalTok{ p}

\CommentTok{\# Calculate the sample size using the formula for margin of error}
\NormalTok{sample\_size }\OperatorTok{=}\NormalTok{ (z\_score }\OperatorTok{**} \DecValTok{2} \OperatorTok{*}\NormalTok{ p }\OperatorTok{*}\NormalTok{ q) }\OperatorTok{/}\NormalTok{ (margin\_of\_error }\OperatorTok{**} \DecValTok{2}\NormalTok{)}

\CommentTok{\# Since sample size must be a whole number, we round up}
\NormalTok{sample\_size }\OperatorTok{=} \BuiltInTok{int}\NormalTok{(sample\_size) }\OperatorTok{+}\NormalTok{ (sample\_size }\OperatorTok{\%} \DecValTok{1} \OperatorTok{\textgreater{}} \DecValTok{0}\NormalTok{)}

\NormalTok{sample\_size}
\end{Highlighting}
\end{Shaded}

\begin{verbatim}
1504
\end{verbatim}

\hypertarget{two-proportions}{%
\subsection{Two Proportions}\label{two-proportions}}

\hypertarget{estimating-a-difference-in-population-proportions}{%
\subsubsection{Estimating a Difference in Population
Proportions}\label{estimating-a-difference-in-population-proportions}}

\[
\begin{align*}
\text{Best Estimate} \pm \text{MoE} \\
\hat{p}_1 - \hat{p}_2 \pm 1.96 \cdot \sqrt{\frac{\hat{p}_1(1-\hat{p}_1)}{n_1} + \frac{\hat{p}_2(1-\hat{p}_2)}{n_2}}
\end{align*}
\]

\begin{Shaded}
\begin{Highlighting}[]
\ImportTok{import}\NormalTok{ numpy }\ImportTok{as}\NormalTok{ np}
\ImportTok{from}\NormalTok{ scipy.stats }\ImportTok{import}\NormalTok{ norm}

\CommentTok{\# Given sample sizes and number of successes}
\NormalTok{n1, x1 }\OperatorTok{=} \DecValTok{988}\NormalTok{, }\DecValTok{543} \CommentTok{\# sample of white children}
\NormalTok{n2, x2 }\OperatorTok{=} \DecValTok{247}\NormalTok{, }\DecValTok{91}  \CommentTok{\# sample of black children}


\CommentTok{\# Calculate the sample proportions}
\NormalTok{p1 }\OperatorTok{=}\NormalTok{ x1 }\OperatorTok{/}\NormalTok{ n1}
\NormalTok{p2 }\OperatorTok{=}\NormalTok{ x2 }\OperatorTok{/}\NormalTok{ n2}

\CommentTok{\# Calculate the standard error for the difference in proportions}
\NormalTok{se }\OperatorTok{=}\NormalTok{ np.sqrt(p1 }\OperatorTok{*}\NormalTok{ (}\DecValTok{1} \OperatorTok{{-}}\NormalTok{ p1) }\OperatorTok{/}\NormalTok{ n1 }\OperatorTok{+}\NormalTok{ p2 }\OperatorTok{*}\NormalTok{ (}\DecValTok{1} \OperatorTok{{-}}\NormalTok{ p2) }\OperatorTok{/}\NormalTok{ n2)}

\CommentTok{\# Calculate the z{-}score for 95\% confidence}
\NormalTok{z }\OperatorTok{=}\NormalTok{ norm.ppf(}\FloatTok{0.975}\NormalTok{)}
\CommentTok{\#z = 1.96}

\CommentTok{\# Calculate the margin of error}
\NormalTok{moE }\OperatorTok{=}\NormalTok{ z }\OperatorTok{*}\NormalTok{ se}

\CommentTok{\# Calculate the confidence interval}
\NormalTok{ci\_lower }\OperatorTok{=}\NormalTok{ (p1 }\OperatorTok{{-}}\NormalTok{ p2) }\OperatorTok{{-}}\NormalTok{ moE}
\NormalTok{ci\_upper }\OperatorTok{=}\NormalTok{ (p1 }\OperatorTok{{-}}\NormalTok{ p2) }\OperatorTok{+}\NormalTok{ moE}
\CommentTok{\#ci\_lower = 0.18 {-} moE}
\CommentTok{\#ci\_upper = 0.18 + moE}

\NormalTok{round\_values([p1 }\OperatorTok{{-}}\NormalTok{ p2, p1, p2, z,moE, ci\_lower, ci\_upper], }\DecValTok{4}\NormalTok{)}
\end{Highlighting}
\end{Shaded}

\begin{verbatim}
[0.1812, 0.5496, 0.3684, 1.96, 0.0677, 0.1135, 0.2489]
\end{verbatim}

Interpreting the Confidence Interval ``range of reasonable values for
our parameter'' With 95\% confidence, the population proportion of
parents with white children who have taken swimming lessons is 11.35 to
24.89\% higher than the population proportion of parents with black
children who have taken swimming lessons.

\hypertarget{population-means}{%
\section{Population Means}\label{population-means}}

\hypertarget{one-mean}{%
\subsection{One Mean}\label{one-mean}}

\[
\bar{x} \pm t^* \left(\frac{s}{\sqrt{n}}\right)
\]

Mean = 82.48 inches

Standard Deviation = 15.06 inches

\(n\) = 25 observations \textgreater{} \(t*\) = 2.064

\[
\begin{align*}
\bar{x} \pm t^* \left( \frac{s}{\sqrt{n}} \right) &= 82.48 \pm 2.064 \left( \frac{15.06}{\sqrt{25}} \right) \\
&= 82.48 \pm 2.064(3.012) \\
&= 82.48 \pm 6.22 \\
& \text{(76.26 inches, 88.70 inches)} 
\end{align*}
\]

\begin{Shaded}
\begin{Highlighting}[]
\ImportTok{from}\NormalTok{ scipy.stats }\ImportTok{import}\NormalTok{ t}

\CommentTok{\# Define your confidence level and degrees of freedom}
\NormalTok{sample\_size }\OperatorTok{=} \DecValTok{25}
\NormalTok{confidence\_level }\OperatorTok{=} \FloatTok{0.95}  \CommentTok{\# for a 95\% confidence interval}
\NormalTok{degrees\_of\_freedom }\OperatorTok{=}\NormalTok{ sample\_size }\OperatorTok{{-}} \DecValTok{1}  \CommentTok{\# degrees of freedom}

\CommentTok{\# Calculate the t{-}score}
\CommentTok{\# The ppf function returns the inverse of the CDF (Cumulative Distribution Function)}
\CommentTok{\# The argument for the ppf function should be 1 minus half of the alpha level (1 {-} alpha/2)}
\CommentTok{\# because the t{-}distribution is symmetric, and we want the cumulative area from {-}t to t.}
\NormalTok{t\_score }\OperatorTok{=}\NormalTok{ t.ppf((}\DecValTok{1} \OperatorTok{+}\NormalTok{ confidence\_level) }\OperatorTok{/} \DecValTok{2}\NormalTok{, degrees\_of\_freedom)}
\CommentTok{\#t\_score = 2.064}

\BuiltInTok{print}\NormalTok{(}\SpecialStringTok{f"t{-}score for }\SpecialCharTok{\{}\NormalTok{confidence\_level}\OperatorTok{*}\DecValTok{100}\SpecialCharTok{\}}\SpecialStringTok{\% confidence level and }\SpecialCharTok{\{}\NormalTok{degrees\_of\_freedom}\SpecialCharTok{\}}\SpecialStringTok{ degrees of freedom: }\SpecialCharTok{\{}\NormalTok{t\_score}\SpecialCharTok{\}}\SpecialStringTok{"}\NormalTok{)}
\end{Highlighting}
\end{Shaded}

\begin{verbatim}
t-score for 95.0% confidence level and 24 degrees of freedom: 2.0638985616280205
\end{verbatim}

\begin{Shaded}
\begin{Highlighting}[]
\CommentTok{\# Given values for the mean, t{-}score, standard deviation, and sample size}
\NormalTok{mean }\OperatorTok{=} \FloatTok{82.48}

\NormalTok{standard\_deviation }\OperatorTok{=} \FloatTok{15.06}


\CommentTok{\# Calculate the margin of error}
\NormalTok{margin\_of\_error }\OperatorTok{=}\NormalTok{ t\_score }\OperatorTok{*}\NormalTok{ (standard\_deviation }\OperatorTok{/}\NormalTok{ (sample\_size }\OperatorTok{**} \FloatTok{0.5}\NormalTok{))}

\CommentTok{\# Calculate the confidence interval}
\NormalTok{lower\_bound }\OperatorTok{=}\NormalTok{ mean }\OperatorTok{{-}}\NormalTok{ margin\_of\_error}
\NormalTok{upper\_bound }\OperatorTok{=}\NormalTok{ mean }\OperatorTok{+}\NormalTok{ margin\_of\_error}

\BuiltInTok{print}\NormalTok{(}\SpecialStringTok{f"Margin of Error: }\SpecialCharTok{\{}\NormalTok{margin\_of\_error}\SpecialCharTok{:.2f\}}\SpecialStringTok{"}\NormalTok{)}
\BuiltInTok{print}\NormalTok{(}\SpecialStringTok{f"Mean (95\% CI): }\SpecialCharTok{\{}\NormalTok{mean}\SpecialCharTok{:.2f\}}\SpecialStringTok{ (}\SpecialCharTok{\{}\NormalTok{lower\_bound}\SpecialCharTok{:.2f\}}\SpecialStringTok{ to }\SpecialCharTok{\{}\NormalTok{upper\_bound}\SpecialCharTok{:.2f\}}\SpecialStringTok{)"}\NormalTok{)}
\end{Highlighting}
\end{Shaded}

\begin{verbatim}
Margin of Error: 6.22
Mean (95% CI): 82.48 (76.26 to 88.70)
\end{verbatim}

\hypertarget{mean-difference-for-paired-data}{%
\subsection{Mean Difference for Paired
Data}\label{mean-difference-for-paired-data}}

\begin{quote}
Difference = older twin - younger twin\\
Difference in Twin Education\\
n = 340 observations\\
Minimum = -3.5 years\\
Maximum = 4 years\\
72.1\% had a difference of 0 years\\
Mean = 0.0838 years\\
Standard Deviation = 0.7627 years
\end{quote}

\begin{quote}
95\% Confidence Interval Calculations\\
Best Estimate + Margin of Error\\
Sample mean difference + ``a few'' . estimated standard error\\
t*\\
t* multiplier comes from a t-distribution with n - 1 degrees of
freedom\\
95\% confidence\\
n= 25 \textgreater{} t* = 2.064\\
n= 1000 \textgreater{} t* = 1.962
\end{quote}

\begin{align*}
\text{Sample mean difference} \ \bar{x}_d \pm t^* \left( \frac{s_d}{\sqrt{n}} \right) & \\
\text{For } n = 25, \ t^* = 2.064 \ \text{, so we have:} & \\
\bar{x}_d \pm 2.064 \left( \frac{s_d}{\sqrt{25}} \right) & \\
\text{If } \bar{x}_d = 82.48 \ \text{and} \ s_d = 15.06, \ \text{then:} & \\
82.48 \pm 2.064 \left( \frac{15.06}{\sqrt{25}} \right) & \\
82.48 \pm 2.064 \times 3.012 & \\
82.48 \pm 6.22 &
\end{align*}

\begin{Shaded}
\begin{Highlighting}[]
\ImportTok{from}\NormalTok{ scipy.stats }\ImportTok{import}\NormalTok{ t}
\ImportTok{import}\NormalTok{ numpy }\ImportTok{as}\NormalTok{ np}

\CommentTok{\# Given values}
\NormalTok{n }\OperatorTok{=} \DecValTok{25}  \CommentTok{\# Sample size}
\NormalTok{confidence\_level }\OperatorTok{=} \FloatTok{0.95}  \CommentTok{\# Confidence level}
\NormalTok{s\_d }\OperatorTok{=} \FloatTok{15.06}  \CommentTok{\# Standard deviation of the differences}
\NormalTok{x\_d\_bar }\OperatorTok{=} \FloatTok{82.48}  \CommentTok{\# Difference in sample means}

\CommentTok{\# Calculate the t{-}score}
\NormalTok{degrees\_of\_freedom }\OperatorTok{=}\NormalTok{ n }\OperatorTok{{-}} \DecValTok{1}
\NormalTok{t\_score }\OperatorTok{=}\NormalTok{ t.ppf((}\DecValTok{1} \OperatorTok{+}\NormalTok{ confidence\_level) }\OperatorTok{/} \DecValTok{2}\NormalTok{, degrees\_of\_freedom)}

\CommentTok{\# Calculate the standard error of the mean difference}
\NormalTok{se\_d }\OperatorTok{=}\NormalTok{ s\_d }\OperatorTok{/}\NormalTok{ np.sqrt(n)}

\CommentTok{\# Calculate the margin of error}
\NormalTok{margin\_of\_error }\OperatorTok{=}\NormalTok{ t\_score }\OperatorTok{*}\NormalTok{ se\_d}

\CommentTok{\# Calculate the confidence interval}
\NormalTok{lower\_bound }\OperatorTok{=}\NormalTok{ x\_d\_bar }\OperatorTok{{-}}\NormalTok{ margin\_of\_error}
\NormalTok{upper\_bound }\OperatorTok{=}\NormalTok{ x\_d\_bar }\OperatorTok{+}\NormalTok{ margin\_of\_error}

\CommentTok{\# Print the results in a publication{-}friendly format}
\BuiltInTok{print}\NormalTok{(}\SpecialStringTok{f"Sample Mean Difference: }\SpecialCharTok{\{}\NormalTok{x\_d\_bar}\SpecialCharTok{:.2f\}}\SpecialStringTok{"}\NormalTok{)}
\BuiltInTok{print}\NormalTok{(}\SpecialStringTok{f"Margin of Error: }\SpecialCharTok{\{}\NormalTok{margin\_of\_error}\SpecialCharTok{:.2f\}}\SpecialStringTok{"}\NormalTok{)}
\BuiltInTok{print}\NormalTok{(}\SpecialStringTok{f"95\% Confidence Interval: (}\SpecialCharTok{\{}\NormalTok{lower\_bound}\SpecialCharTok{:.2f\}}\SpecialStringTok{, }\SpecialCharTok{\{}\NormalTok{upper\_bound}\SpecialCharTok{:.2f\}}\SpecialStringTok{)"}\NormalTok{)}
\end{Highlighting}
\end{Shaded}

\begin{verbatim}
Sample Mean Difference: 82.48
Margin of Error: 6.22
95% Confidence Interval: (76.26, 88.70)
\end{verbatim}

\begin{align*}
\text{Mean Difference Confidence Interval} \\
\text{Mean} &= 0.084 \text{ years} \\
\text{Standard Deviation} &= 0.76 \text{ years} \\
n &= 340 \text{ observations} \rightarrow t^* = 1.967 \\
\end{align*}

\begin{align*}
\bar{x}_d \pm t^* \left( \frac{s_d}{\sqrt{n}} \right) &= 0.084 \pm 1.967 \left( \frac{0.76}{\sqrt{340}} \right) \\
&= 0.084 \pm 1.967 (0.04) \\
&= 0.084 \pm 0.0814 \\
&= (0.0026, 0.1654) \text{ years}
\end{align*}

\begin{Shaded}
\begin{Highlighting}[]
\ImportTok{from}\NormalTok{ scipy.stats }\ImportTok{import}\NormalTok{ t}
\ImportTok{import}\NormalTok{ numpy }\ImportTok{as}\NormalTok{ np}

\CommentTok{\# Given values from the image}
\NormalTok{mean\_diff }\OperatorTok{=} \FloatTok{0.084}  \CommentTok{\# Mean difference (x̄\_d)}
\NormalTok{std\_dev\_diff }\OperatorTok{=} \FloatTok{0.76}  \CommentTok{\# Standard deviation of differences (s\_d)}
\NormalTok{n }\OperatorTok{=} \DecValTok{340}  \CommentTok{\# Sample size (n)}
\NormalTok{confidence\_level }\OperatorTok{=} \FloatTok{0.95}  \CommentTok{\# Confidence level for 95\%}

\CommentTok{\# Calculate the t{-}score using the degrees of freedom (n {-} 1)}
\NormalTok{degrees\_of\_freedom }\OperatorTok{=}\NormalTok{ n }\OperatorTok{{-}} \DecValTok{1}
\CommentTok{\#t\_score = t.ppf((1 + confidence\_level) / 2, degrees\_of\_freedom)}
\NormalTok{t\_score }\OperatorTok{=} \FloatTok{1.967}

\CommentTok{\# Calculate the standard error of the mean difference}
\NormalTok{standard\_error\_diff }\OperatorTok{=}\NormalTok{ std\_dev\_diff }\OperatorTok{/}\NormalTok{ np.sqrt(n)}

\CommentTok{\# Calculate the margin of error}
\NormalTok{margin\_of\_error }\OperatorTok{=}\NormalTok{ t\_score }\OperatorTok{*}\NormalTok{ standard\_error\_diff}

\CommentTok{\# Calculate the confidence interval}
\NormalTok{lower\_bound }\OperatorTok{=}\NormalTok{ mean\_diff }\OperatorTok{{-}}\NormalTok{ margin\_of\_error}
\NormalTok{upper\_bound }\OperatorTok{=}\NormalTok{ mean\_diff }\OperatorTok{+}\NormalTok{ margin\_of\_error}

\CommentTok{\# Print the results}
\BuiltInTok{print}\NormalTok{(}\SpecialStringTok{f"Sample Mean Difference: }\SpecialCharTok{\{}\NormalTok{mean\_diff}\SpecialCharTok{\}}\SpecialStringTok{ years"}\NormalTok{)}
\BuiltInTok{print}\NormalTok{(}\SpecialStringTok{f"Margin of Error: }\SpecialCharTok{\{}\NormalTok{margin\_of\_error}\SpecialCharTok{:.4f\}}\SpecialStringTok{ years"}\NormalTok{)}
\BuiltInTok{print}\NormalTok{(}\SpecialStringTok{f"95\% Confidence Interval: (}\SpecialCharTok{\{}\NormalTok{lower\_bound}\SpecialCharTok{:.4f\}}\SpecialStringTok{, }\SpecialCharTok{\{}\NormalTok{upper\_bound}\SpecialCharTok{:.4f\}}\SpecialStringTok{) years"}\NormalTok{)}
\end{Highlighting}
\end{Shaded}

\begin{verbatim}
Sample Mean Difference: 0.084 years
Margin of Error: 0.0811 years
95% Confidence Interval: (0.0029, 0.1651) years
\end{verbatim}

With 95\% confidence, the population mean difference of the older twin's
less the younger twin's self-reported education is estimated to be
between lower\_bound years and upper\_bound years.

\hypertarget{using-scipy-and-statsmodels}{%
\subsubsection{Using SciPy and
StatsModels}\label{using-scipy-and-statsmodels}}

\begin{Shaded}
\begin{Highlighting}[]
\ImportTok{import}\NormalTok{ numpy }\ImportTok{as}\NormalTok{ np}
\ImportTok{import}\NormalTok{ scipy.stats }\ImportTok{as}\NormalTok{ stats}
\ImportTok{import}\NormalTok{ statsmodels.stats.api }\ImportTok{as}\NormalTok{ sms}

\CommentTok{\# Given values from the image}
\NormalTok{mean\_diff }\OperatorTok{=} \FloatTok{0.084}  \CommentTok{\# Mean difference (x̄\_d)}
\NormalTok{std\_dev\_diff }\OperatorTok{=} \FloatTok{0.76}  \CommentTok{\# Standard deviation of differences (s\_d)}
\NormalTok{n }\OperatorTok{=} \DecValTok{340}  \CommentTok{\# Sample size (n)}
\NormalTok{confidence\_level }\OperatorTok{=} \FloatTok{0.95}  \CommentTok{\# Confidence level for 95\%}

\CommentTok{\# Calculate the standard error of the mean difference}
\NormalTok{standard\_error\_diff }\OperatorTok{=}\NormalTok{ std\_dev\_diff }\OperatorTok{/}\NormalTok{ np.sqrt(n)}

\CommentTok{\# Using SciPy to calculate the confidence interval}
\NormalTok{ci\_low, ci\_up }\OperatorTok{=}\NormalTok{ stats.t.interval(confidence\_level, df}\OperatorTok{=}\NormalTok{n}\OperatorTok{{-}}\DecValTok{1}\NormalTok{, loc}\OperatorTok{=}\NormalTok{mean\_diff, scale}\OperatorTok{=}\NormalTok{standard\_error\_diff)}

\CommentTok{\# Using StatsModels to calculate the confidence interval}
\NormalTok{cm }\OperatorTok{=}\NormalTok{ sms.CompareMeans(sms.DescrStatsW(np.random.normal(mean\_diff, std\_dev\_diff, n)), }
\NormalTok{                      sms.DescrStatsW(np.random.normal(mean\_diff, std\_dev\_diff, n)))}
\NormalTok{ci\_low\_sms, ci\_up\_sms }\OperatorTok{=}\NormalTok{ cm.tconfint\_diff(usevar}\OperatorTok{=}\StringTok{\textquotesingle{}unequal\textquotesingle{}}\NormalTok{)}

\BuiltInTok{print}\NormalTok{(}\SpecialStringTok{f"Confidence Interval using SciPy: (}\SpecialCharTok{\{}\NormalTok{ci\_low}\SpecialCharTok{:.4f\}}\SpecialStringTok{, }\SpecialCharTok{\{}\NormalTok{ci\_up}\SpecialCharTok{:.4f\}}\SpecialStringTok{) years"}\NormalTok{)}
\BuiltInTok{print}\NormalTok{(}\SpecialStringTok{f"Confidence Interval using StatsModels: (}\SpecialCharTok{\{}\NormalTok{ci\_low\_sms}\SpecialCharTok{:.4f\}}\SpecialStringTok{, }\SpecialCharTok{\{}\NormalTok{ci\_up\_sms}\SpecialCharTok{:.4f\}}\SpecialStringTok{) years"}\NormalTok{)}
\end{Highlighting}
\end{Shaded}

\begin{verbatim}
Confidence Interval using SciPy: (0.0029, 0.1651) years
Confidence Interval using StatsModels: (-0.1436, 0.0927) years
\end{verbatim}

The SciPy and StatsModels libraries might give different confidence
intervals (CIs) due to the way they handle the data and calculations
internally, particularly if they're using different assumptions or
default parameters in their calculations. Here are a few reasons why
differences might arise:

\begin{enumerate}
\def\labelenumi{\arabic{enumi}.}
\item
  \textbf{Underlying Assumptions}: StatsModels might use a different
  method to estimate the variance (e.g., assuming equal or unequal
  variances in two samples when calculating confidence intervals for
  mean differences).
\item
  \textbf{Data Handling}: The snippet provided uses simulated data for
  StatsModels, which could introduce variability. Normally, you would
  use actual data rather than simulated data to ensure consistency.
\item
  \textbf{Defaults for Degrees of Freedom}: The method of calculating
  degrees of freedom may vary. Some methods adjust degrees of freedom
  based on sample size and variance, particularly in the case of two
  independent samples with potentially unequal variances.
\item
  \textbf{Numerical Precision}: Differences in numerical precision can
  result from the specific algorithms used to calculate the t-scores and
  the standard error within each package.
\item
  \textbf{Confidence Interval Estimation}: Some statistical packages
  might use a slightly different formula or method to calculate the
  confidence interval. For example, StatsModels provides several ways to
  calculate the confidence interval for the mean difference, including
  assuming equal variances
  (\texttt{usevar=\textquotesingle{}pooled\textquotesingle{}}) or not
  (\texttt{usevar=\textquotesingle{}unequal\textquotesingle{}}).
\end{enumerate}

To ensure consistency, it's important to use the same dataset and
understand the default behaviors and assumptions of the statistical
functions you are using. If precise control over the calculation method
is required, you should manually specify the parameters to match your
statistical model's assumptions.

\hypertarget{estimating-a-difference-in-population-means-for-independent-groups}{%
\subsection{Estimating a Difference in Population Means for Independent
Groups}\label{estimating-a-difference-in-population-means-for-independent-groups}}

Here are the steps to estimate a difference in population means with
confidence for two independent groups:

\begin{enumerate}
\def\labelenumi{\arabic{enumi}.}
\tightlist
\item
  \textbf{Identify the Populations and Parameter of Interest}:

  \begin{itemize}
  \tightlist
  \item
    Specify the populations you are comparing (e.g., Mexican-American
    adults aged 18 to 29 living in the United States).
  \item
    Define the parameter of interest (e.g., BMI measured in kilograms
    per meter squared).
  \item
    Use symbols ( \mu\_1 ) and ( \mu\_2 ) to represent the means of the
    two populations, and focus on the difference ( \mu\_1 - \mu\_2 ).
  \end{itemize}
\item
  \textbf{Collect Samples and Calculate Descriptive Statistics}:

  \begin{itemize}
  \tightlist
  \item
    Sample the two populations separately according to the
    characteristic of interest (e.g., gender from the NHANES dataset).
  \item
    Calculate the mean, standard deviation, minimum, maximum, and sample
    size for both sub-groups.
  \end{itemize}
\item
  \textbf{Understand the Distribution of Sample Mean Differences}:

  \begin{itemize}
  \tightlist
  \item
    Recognize that the sample mean will vary from sample to sample,
    creating a sampling distribution.
  \item
    Note that if both population responses are approximately normal or
    sample sizes are large, the distribution of the difference in sample
    means will be approximately normal.
  \end{itemize}
\item
  \textbf{Estimate the Standard Error}:

  \begin{itemize}
  \tightlist
  \item
    Acknowledge that we often don't know true population variances, so
    we estimate them using sample variances.
  \item
    Use the formula for standard error involving the sample standard
    deviations and sizes of both groups.
  \end{itemize}
\item
  \textbf{Construct the Confidence Interval}:

  \begin{itemize}
  \tightlist
  \item
    Start with the difference in sample means as the best estimate.
  \item
    Add and subtract the margin of error, which is a multiple of the
    estimated standard error.
  \item
    Use a t-score multiplier from the t-distribution corresponding to
    the desired confidence level.
  \end{itemize}
\item
  \textbf{Choose the Correct Approach---Pooled or Unpooled}:

  \begin{itemize}
  \tightlist
  \item
    If population variances are assumed to be equal, use the pooled
    approach; otherwise, use the unpooled approach.
  \item
    For the pooled approach, combine the variances of the two samples.
  \item
    For the unpooled approach, keep the variances separate.
  \end{itemize}
\item
  \textbf{Interpret the Confidence Interval}:

  \begin{itemize}
  \tightlist
  \item
    Understand that the interval provides a range of reasonable values
    for the parameter of interest.
  \item
    The confidence level indicates the proportion of such intervals that
    would contain the true population mean difference if the procedure
    were repeatedly applied.
  \end{itemize}
\item
  \textbf{Verify Assumptions}:

  \begin{itemize}
  \tightlist
  \item
    Ensure both samples are simple random samples and independent of
    each other.
  \item
    Check for normality using graphical methods like histograms and Q-Q
    plots.
  \item
    Apply the Central Limit Theorem if sample sizes are large to justify
    normality.
  \end{itemize}
\item
  \textbf{Decide on Variance Equality}:

  \begin{itemize}
  \tightlist
  \item
    Inspect sample variances and interquartile ranges to assess whether
    population variances can be assumed equal.
  \item
    Choose the pooled method if variances are similar, or the unpooled
    method if not.
  \end{itemize}
\item
  \textbf{Calculate and Interpret the Final Confidence Interval}:

  \begin{itemize}
  \tightlist
  \item
    Use the chosen method to calculate the confidence interval for the
    difference in population means.
  \item
    Communicate the results, stating the range and confidence level
    clearly.
  \end{itemize}
\end{enumerate}

Each of these steps provides the framework for estimating the difference
in population means with confidence, ensuring a robust statistical
analysis of two independent groups.

The image you've uploaded appears to contain a slide from a presentation
that outlines a research question. It reads:

\begin{center}\rule{0.5\linewidth}{0.5pt}\end{center}

\hypertarget{research-question}{%
\subsubsection{Research Question}\label{research-question}}

Considering Mexican-American adults (ages 18 - 29) living in the United
States, do males and females differ significantly in mean Body Mass
Index (BMI)?

\begin{itemize}
\tightlist
\item
  Population: Mexican-American adults (ages 18 - 29) in the U.S.
\item
  Parameter of Interest (μ1 - μ2): Body Mass Index or BMI (kg/m²)
\end{itemize}

\begin{center}\rule{0.5\linewidth}{0.5pt}\end{center}

To address this research question, we would typically follow a series of
steps:

\begin{enumerate}
\def\labelenumi{\arabic{enumi}.}
\item
  \textbf{Hypothesis Formulation}: Based on the parameter of interest,
  which is the difference in mean BMI between males (μ1) and females
  (μ2), we would set up the null hypothesis (H0: μ1 - μ2 = 0) indicating
  no difference and the alternative hypothesis (H1: μ1 - μ2 ≠ 0)
  indicating a significant difference.
\item
  \textbf{Data Collection}: Gather a representative sample of
  Mexican-American adults, ensuring a balanced representation of males
  and females aged 18-29.
\item
  \textbf{Data Analysis}: Calculate the mean BMI for both groups and
  perform a hypothesis test, such as an independent t-test, to evaluate
  the difference in means.
\item
  \textbf{Result Interpretation}: Determine if the observed difference
  is statistically significant based on the p-value and the
  pre-determined level of significance (commonly α = 0.05).
\item
  \textbf{Conclusion}: Draw conclusions from the statistical analysis
  and relate them back to the research question.
\end{enumerate}

If we have access to a dataset containing the relevant information, I
can assist with the Python code to perform the necessary statistical
analysis. Please let me know if you would like to proceed with this, or
if there is another aspect of this research question you would like to
explore.

\begin{Shaded}
\begin{Highlighting}[]
\CommentTok{\# Step 1: Import necessary libraries}
\ImportTok{import}\NormalTok{ pandas }\ImportTok{as}\NormalTok{ pd}
\ImportTok{from}\NormalTok{ scipy }\ImportTok{import}\NormalTok{ stats}

\CommentTok{\# Step 2: Load your dataset}
\CommentTok{\# Assuming you have a CSV file named \textquotesingle{}data.csv\textquotesingle{} with columns \textquotesingle{}age\textquotesingle{}, \textquotesingle{}gender\textquotesingle{}, \textquotesingle{}bmi\textquotesingle{}}
\CommentTok{\# data = pd.read\_csv(\textquotesingle{}path\_to\_your\_data.csv\textquotesingle{})}

\CommentTok{\# For demonstration purposes, let\textquotesingle{}s create a mock DataFrame}
\CommentTok{\# Replace this mock data with your actual data}
\NormalTok{data }\OperatorTok{=}\NormalTok{ pd.DataFrame(\{}
    \StringTok{\textquotesingle{}age\textquotesingle{}}\NormalTok{: [}\DecValTok{18}\NormalTok{, }\DecValTok{20}\NormalTok{, }\DecValTok{29}\NormalTok{] }\OperatorTok{*} \DecValTok{20}\NormalTok{,  }\CommentTok{\# Mock ages}
    \StringTok{\textquotesingle{}gender\textquotesingle{}}\NormalTok{: [}\StringTok{\textquotesingle{}male\textquotesingle{}}\NormalTok{, }\StringTok{\textquotesingle{}female\textquotesingle{}}\NormalTok{] }\OperatorTok{*} \DecValTok{30}\NormalTok{,  }\CommentTok{\# Mock genders}
    \StringTok{\textquotesingle{}bmi\textquotesingle{}}\NormalTok{: np.random.normal(}\DecValTok{25}\NormalTok{, }\DecValTok{5}\NormalTok{, }\DecValTok{60}\NormalTok{)  }\CommentTok{\# Mock BMIs with a mean of 25 and std dev of 5}
\NormalTok{\})}

\CommentTok{\# Step 3: Clean and format the data if necessary}
\CommentTok{\# Ensure that the \textquotesingle{}gender\textquotesingle{} column is categorized properly}
\NormalTok{data[}\StringTok{\textquotesingle{}gender\textquotesingle{}}\NormalTok{] }\OperatorTok{=}\NormalTok{ data[}\StringTok{\textquotesingle{}gender\textquotesingle{}}\NormalTok{].astype(}\StringTok{\textquotesingle{}category\textquotesingle{}}\NormalTok{)}

\CommentTok{\# Filter the dataset for the age range 18{-}29}
\NormalTok{data }\OperatorTok{=}\NormalTok{ data[(data[}\StringTok{\textquotesingle{}age\textquotesingle{}}\NormalTok{] }\OperatorTok{\textgreater{}=} \DecValTok{18}\NormalTok{) }\OperatorTok{\&}\NormalTok{ (data[}\StringTok{\textquotesingle{}age\textquotesingle{}}\NormalTok{] }\OperatorTok{\textless{}=} \DecValTok{29}\NormalTok{)]}

\CommentTok{\# Step 4: Separate the data into two groups}
\NormalTok{male\_bmi }\OperatorTok{=}\NormalTok{ data[data[}\StringTok{\textquotesingle{}gender\textquotesingle{}}\NormalTok{] }\OperatorTok{==} \StringTok{\textquotesingle{}male\textquotesingle{}}\NormalTok{][}\StringTok{\textquotesingle{}bmi\textquotesingle{}}\NormalTok{]}
\NormalTok{female\_bmi }\OperatorTok{=}\NormalTok{ data[data[}\StringTok{\textquotesingle{}gender\textquotesingle{}}\NormalTok{] }\OperatorTok{==} \StringTok{\textquotesingle{}female\textquotesingle{}}\NormalTok{][}\StringTok{\textquotesingle{}bmi\textquotesingle{}}\NormalTok{]}

\CommentTok{\# Step 5: Calculate the mean BMI for each group}
\NormalTok{mean\_male\_bmi }\OperatorTok{=}\NormalTok{ male\_bmi.mean()}
\NormalTok{mean\_female\_bmi }\OperatorTok{=}\NormalTok{ female\_bmi.mean()}

\CommentTok{\# Step 6: Perform an independent t{-}test}
\NormalTok{t\_stat, p\_value }\OperatorTok{=}\NormalTok{ stats.ttest\_ind(male\_bmi, female\_bmi)}

\CommentTok{\# Output the results}
\BuiltInTok{print}\NormalTok{(}\SpecialStringTok{f"Mean BMI for males: }\SpecialCharTok{\{}\NormalTok{mean\_male\_bmi}\SpecialCharTok{:.2f\}}\SpecialStringTok{"}\NormalTok{)}
\BuiltInTok{print}\NormalTok{(}\SpecialStringTok{f"Mean BMI for females: }\SpecialCharTok{\{}\NormalTok{mean\_female\_bmi}\SpecialCharTok{:.2f\}}\SpecialStringTok{"}\NormalTok{)}
\BuiltInTok{print}\NormalTok{(}\SpecialStringTok{f"T{-}statistic: }\SpecialCharTok{\{}\NormalTok{t\_stat}\SpecialCharTok{:.2f\}}\SpecialStringTok{"}\NormalTok{)}
\BuiltInTok{print}\NormalTok{(}\SpecialStringTok{f"P{-}value: }\SpecialCharTok{\{}\NormalTok{p\_value}\SpecialCharTok{:.4f\}}\SpecialStringTok{"}\NormalTok{)}

\CommentTok{\# Based on the p{-}value, determine if the result is statistically significant}
\NormalTok{alpha }\OperatorTok{=} \FloatTok{0.05}  \CommentTok{\# Common significance level}
\ControlFlowTok{if}\NormalTok{ p\_value }\OperatorTok{\textless{}}\NormalTok{ alpha:}
    \BuiltInTok{print}\NormalTok{(}\StringTok{"We reject the null hypothesis, suggesting a significant difference in BMI between genders."}\NormalTok{)}
\ControlFlowTok{else}\NormalTok{:}
    \BuiltInTok{print}\NormalTok{(}\StringTok{"We fail to reject the null hypothesis, suggesting no significant difference in BMI between genders."}\NormalTok{)}
\end{Highlighting}
\end{Shaded}

\begin{verbatim}
Mean BMI for males: 23.93
Mean BMI for females: 24.59
T-statistic: -0.52
P-value: 0.6085
We fail to reject the null hypothesis, suggesting no significant difference in BMI between genders.
\end{verbatim}

\begin{Shaded}
\begin{Highlighting}[]
\ImportTok{import}\NormalTok{ matplotlib.pyplot }\ImportTok{as}\NormalTok{ plt}
\ImportTok{import}\NormalTok{ numpy }\ImportTok{as}\NormalTok{ np}
\ImportTok{import}\NormalTok{ pandas }\ImportTok{as}\NormalTok{ pd}

\CommentTok{\# Assuming you have a DataFrame \textquotesingle{}data\textquotesingle{} with \textquotesingle{}gender\textquotesingle{} and \textquotesingle{}bmi\textquotesingle{} columns}

\CommentTok{\# Create a box plot}
\NormalTok{plt.figure(figsize}\OperatorTok{=}\NormalTok{(}\DecValTok{8}\NormalTok{, }\DecValTok{6}\NormalTok{))  }\CommentTok{\# Set the figure size}
\NormalTok{boxplot }\OperatorTok{=}\NormalTok{ data.boxplot(column}\OperatorTok{=}\StringTok{\textquotesingle{}bmi\textquotesingle{}}\NormalTok{, by}\OperatorTok{=}\StringTok{\textquotesingle{}gender\textquotesingle{}}\NormalTok{)}
\NormalTok{plt.title(}\StringTok{\textquotesingle{}BMI Distribution by Gender\textquotesingle{}}\NormalTok{)}
\NormalTok{plt.suptitle(}\StringTok{\textquotesingle{}\textquotesingle{}}\NormalTok{)  }\CommentTok{\# Suppress the default title}
\NormalTok{plt.xlabel(}\StringTok{\textquotesingle{}Gender\textquotesingle{}}\NormalTok{)}
\NormalTok{plt.ylabel(}\StringTok{\textquotesingle{}BMI\textquotesingle{}}\NormalTok{)}

\CommentTok{\# Show the plot}
\NormalTok{plt.show()}
\end{Highlighting}
\end{Shaded}

\begin{verbatim}
<Figure size 800x600 with 0 Axes>
\end{verbatim}

\begin{figure}[H]

{\centering \includegraphics{PyStats_files/figure-pdf/cell-19-output-2.png}

}

\end{figure}

\begin{Shaded}
\begin{Highlighting}[]
\ImportTok{import}\NormalTok{ seaborn }\ImportTok{as}\NormalTok{ sns}

\CommentTok{\# Create a box plot with Seaborn}
\NormalTok{plt.figure(figsize}\OperatorTok{=}\NormalTok{(}\DecValTok{8}\NormalTok{, }\DecValTok{6}\NormalTok{))}
\NormalTok{sns.boxplot(x}\OperatorTok{=}\StringTok{\textquotesingle{}gender\textquotesingle{}}\NormalTok{, y}\OperatorTok{=}\StringTok{\textquotesingle{}bmi\textquotesingle{}}\NormalTok{, data}\OperatorTok{=}\NormalTok{data)}
\NormalTok{plt.title(}\StringTok{\textquotesingle{}BMI Distribution by Gender\textquotesingle{}}\NormalTok{)}
\NormalTok{plt.xlabel(}\StringTok{\textquotesingle{}Gender\textquotesingle{}}\NormalTok{)}
\NormalTok{plt.ylabel(}\StringTok{\textquotesingle{}BMI\textquotesingle{}}\NormalTok{)}

\CommentTok{\# Show the plot}
\NormalTok{plt.show()}
\end{Highlighting}
\end{Shaded}

\begin{figure}[H]

{\centering \includegraphics{PyStats_files/figure-pdf/cell-20-output-1.png}

}

\end{figure}

\begin{Shaded}
\begin{Highlighting}[]
\ImportTok{import}\NormalTok{ numpy }\ImportTok{as}\NormalTok{ np}
\ImportTok{import}\NormalTok{ matplotlib.pyplot }\ImportTok{as}\NormalTok{ plt}
\ImportTok{from}\NormalTok{ scipy.stats }\ImportTok{import}\NormalTok{ norm}

\CommentTok{\# Parameters for the sampling distribution}
\CommentTok{\# For demonstration, we assume the following:}
\CommentTok{\# The difference in population means is 0 under the null hypothesis}
\NormalTok{mu\_diff }\OperatorTok{=} \DecValTok{0}
\CommentTok{\# Standard deviation of the sampling distribution, assumed to be 1 for this example}
\NormalTok{sigma\_diff }\OperatorTok{=} \DecValTok{1}
\CommentTok{\# Sample size for illustration}
\NormalTok{sample\_size }\OperatorTok{=} \DecValTok{1000}

\CommentTok{\# Generate a sampling distribution of the difference in means}
\NormalTok{sample\_diff\_means }\OperatorTok{=}\NormalTok{ np.random.normal(mu\_diff, sigma\_diff, sample\_size)}

\CommentTok{\# Plot the distribution}
\NormalTok{plt.figure(figsize}\OperatorTok{=}\NormalTok{(}\DecValTok{10}\NormalTok{, }\DecValTok{6}\NormalTok{))}
\NormalTok{plt.hist(sample\_diff\_means, bins}\OperatorTok{=}\DecValTok{30}\NormalTok{, density}\OperatorTok{=}\VariableTok{True}\NormalTok{, alpha}\OperatorTok{=}\FloatTok{0.6}\NormalTok{, color}\OperatorTok{=}\StringTok{\textquotesingle{}b\textquotesingle{}}\NormalTok{)}

\CommentTok{\# Plot the normal distribution curve}
\NormalTok{xmin, xmax }\OperatorTok{=}\NormalTok{ plt.xlim()}
\NormalTok{x }\OperatorTok{=}\NormalTok{ np.linspace(xmin, xmax, }\DecValTok{100}\NormalTok{)}
\NormalTok{p }\OperatorTok{=}\NormalTok{ norm.pdf(x, mu\_diff, sigma\_diff)}
\NormalTok{plt.plot(x, p, }\StringTok{\textquotesingle{}k\textquotesingle{}}\NormalTok{, linewidth}\OperatorTok{=}\DecValTok{2}\NormalTok{)}

\NormalTok{title }\OperatorTok{=} \StringTok{"Sampling Distribution of the Difference in Two Independent Sample Means"}
\NormalTok{plt.title(title)}
\NormalTok{plt.xlabel(}\StringTok{\textquotesingle{}Difference in sample means\textquotesingle{}}\NormalTok{)}
\NormalTok{plt.ylabel(}\StringTok{\textquotesingle{}Frequency\textquotesingle{}}\NormalTok{)}
\NormalTok{plt.show()}
\end{Highlighting}
\end{Shaded}

\begin{figure}[H]

{\centering \includegraphics{PyStats_files/figure-pdf/cell-21-output-1.png}

}

\end{figure}

The graph above illustrates the sampling distribution of the difference
in two independent sample means. It shows a normal distribution centered
around zero, which is what we'd expect under the null hypothesis that
there is no difference between the two population means
(\(\mu_1 - \mu_2 = 0\)).

The histogram represents the frequency of various differences observed
in the sample means, while the black curve is the probability density
function of a normal distribution with the mean (\(\mu_{diff}\)) set to
0 and a standard deviation (\(\sigma_{diff}\)) assumed to be 1 for this
example. This visual representation helps us understand the variability
we might expect by chance when estimating the difference between two
means from independent samples.

In the context of a real-world application, such as comparing BMI
between genders, you would calculate the actual standard deviation of
the sampling distribution based on the sample data and the sample sizes
of the groups being compared. This would then inform your hypothesis
testing, allowing you to determine the likelihood of observing the
sample mean difference if the null hypothesis were true.

The image you've uploaded depicts a concept from inferential statistics
known as the ``Sampling Distribution of the Difference in Two
(Independent) Sample Means''. This concept is pivotal when comparing the
means from two independent groups to determine if there is a
statistically significant difference between them.

The key points outlined in the image are:

\begin{enumerate}
\def\labelenumi{\arabic{enumi}.}
\tightlist
\item
  If both populations of responses have distributions that are
  approximately normal, or if the sample sizes are large enough, then
  the distribution of the difference in sample means is also
  approximately normal.
\item
  The center of this sampling distribution is at the difference in
  population means \(\mu_1 - \mu_2\).
\item
  The distribution allows for the calculation of the probability of
  observing a sample mean difference as extreme as, or more extreme
  than, the one observed.
\end{enumerate}

In the context of the previous discussion about BMI across genders, the
sampling distribution would be used to estimate the likelihood of
observing the difference in mean BMIs between males and females under
the null hypothesis that there is no true difference in the population
means.

Calculation of the standard error for the sampling distribution of the
difference between two independent sample means. The standard error is a
measure of the variability of the sampling distribution and is crucial
when conducting hypothesis tests or constructing confidence intervals.

The standard error formula for the difference between two independent
means is provided in the image as:

\(SE = \sqrt{\frac{{\sigma_1}^2}{n_1} + \frac{{\sigma_2}^2}{n_2}}\)

Where: - \(\sigma_1^2\) and \(\sigma_2^2\) are the population variances
of the first and second groups, respectively. - \(n_1\) and \(n_2\) are
the sample sizes of the first and second groups, respectively.

Since population variances are rarely known, we use sample variances
\(s_1^2\) and \(s_2^2\) to estimate the standard error:

\(Estimated\:SE = \sqrt{\frac{{s_1}^2}{n_1} + \frac{{s_2}^2}{n_2}}\)

This estimated standard error is used to calculate test statistics like
the t-statistic in an independent samples t-test, or to construct
confidence intervals for the difference between the two means.

Would you like to see how to calculate this estimated standard error
using Python, given sample data?

\hypertarget{end}{%
\section{End}\label{end}}



\end{document}
